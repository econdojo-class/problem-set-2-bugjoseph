%------------%
%  Preamble  %
%------------%

\documentclass[final,11pt]{article}
\usepackage[paperwidth=9.0in, top=1.2in, bottom=1.2in, left=1.2in, right=1.2in]{geometry}
\usepackage{amsmath}
\usepackage{color}
\usepackage{multirow}
\usepackage{setspace}
\usepackage{fancyhdr}
\usepackage{longtable}
\usepackage{array}
\usepackage{booktabs}
\usepackage{mathpazo}
\usepackage{graphicx}
\usepackage{threeparttable}
\usepackage{eurosym}
\usepackage[colorlinks, linkcolor=blue, anchorcolor=blue, citecolor=blue]{hyperref}

\renewcommand{\headrulewidth}{0pt}
\setlength{\arraycolsep}{10pt}
\setlength\headheight{0.5cm}
\setlength\headsep{0.8cm}
\setlength\footskip{1.0cm}
\setlength{\parindent}{0em}
\pagestyle{fancy}
\chead{\textcolor[rgb]{0.5,0.5,0.5}{\sc Spring 2025: ECON 6100}}

%------------%
%  Document  %
%------------%

\begin{document}
\thispagestyle{empty}
\begin{spacing}{1.25}

\textbf{Your Name:  JOSEPH AGOR  \hfill Problem Set 2, Due: Mar. 18, 2025}\\

(1) Use the probability integral transformation method to simulate from the distribution
\begin{gather}
    f(x) = 
    \begin{cases}
        \frac{2}{a^2}x,  & \text{if }0\leq x\leq a \\
        0, & \text{otherwise}
    \end{cases}
\end{gather}
where $a>0$. Set a value for $a$, simulate various sample sizes, and compare results to the true distribution.


\section*{Question 1: Probability Integral Transformation}
Given the probability density function:
\begin{equation}
 f(x) = \begin{cases}
        \frac{2}{a^2} x, & 0 \leq x \leq a \\
        0, & \text{otherwise}
    \end{cases}
\end{equation}
We use the inverse transform method:
\begin{equation}
    F(x) = \int_0^x \frac{2}{a^2} t dt = \frac{x^2}{a^2}.
\end{equation}
Solving for $x$ in terms of $U \sim U(0,1)$:
\begin{equation}
    x = a \sqrt{U}.
\end{equation}
We simulate samples for different sizes and compare them with the theoretical distribution. 

\begin{center}
    \includegraphics[width=0.8\textwidth]{triangular_distribution.png}
\end{center}

\newpage 

(2) Generate samples from the distribution
\begin{gather}
    f(x)=\frac{2}{3}e^{-2x}+2e^{-3x}
\end{gather}
using the finite mixture approach.


\section*{Question 2: Mixture of Exponentials}
The given density function is:
\begin{equation}
    f(x) = \frac{2}{3} e^{-2x} + 2e^{-3x}.
\end{equation}
We use the finite mixture approach where each component is selected with equal probability. The sampling method follows:
\begin{itemize}
    \item Generate $U \sim U(0,1)$.
    \item If $U < 0.5$, sample $X \sim \text{Exp}(2)$, else sample $X \sim \text{Exp}(3)$.
\end{itemize}
\begin{center}
    \includegraphics[width=0.8\textwidth]{mixture_exponentials.png}
\end{center}

\newpage

(3) Draw 500 observations from Beta$(3,3)$ using the accept-reject algorithm. Compute the mean and variance of the sample and compare them to the true values.


\section*{Question 3: Beta(3,3) using Accept-Reject}
Using $g(x) = U(0,1)$ as the proposal distribution, we have:
\begin{equation}
    f(x) = \frac{\Gamma(6)}{\Gamma(3)\Gamma(3)} x^2 (1-x)^2.
\end{equation}
The constant $c$ is chosen such that:
\begin{equation}
    c \geq \sup_x \frac{f(x)}{g(x)} = 1.5.
\end{equation}
The accept-reject algorithm follows:
\begin{itemize}
    \item Generate $Y \sim U(0,1)$.
    \item Generate $U \sim U(0,1)$.
    \item Accept $Y$ if $U \leq \frac{f(Y)}{c g(Y)}$.
\end{itemize}
The empirical mean and variance are compared with theoretical values.

\begin{center}
    \includegraphics[width=0.8\textwidth]{beta_distribution.png}
\end{center}

Empirical Mean: \texttt{\textbf{0.508045063498666}}, Theoretical Mean: \texttt{0.5} \\
Empirical Variance: \texttt{\textbf{0.03895205175094688}}, Theoretical Variance: \texttt{0.0278}




